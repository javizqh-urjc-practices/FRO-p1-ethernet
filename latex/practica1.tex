\documentclass[12pt, a4paper]{report}
\usepackage{graphicx} %LaTeX package to import graphics
\usepackage{enumitem}
\usepackage{geometry}
\geometry{lmargin=30mm}
\usepackage[export]{adjustbox}
\usepackage{titlesec}
\titleformat{\chapter}{\normalfont\huge}{\thechapter}{20pt}{\huge\bf}
\graphicspath{{images/}} %configuring the graphicx package
\title{Practica 1}
\author{Javier Izquierdo Hernández}
\date{\today}
\begin{document}
	\begin{titlepage}
		\centering
		{\includegraphics[width=0.3\textwidth]{logo}\par}
		\vspace{1cm}
		{\bfseries\LARGE Universidad Rey Juan Carlos \par}
		\vspace{1cm}
		{\scshape\Large E.T.S. Ingeniería de Telecomunicación \par}
		\vspace{3cm}
		{\scshape\Huge Fundamentos de redes de ordenadores \par}
		\vspace{3cm}
		{\itshape\Large Práctica 1 \par}
		\vfill
		{\Large Autor: \par}
		{\Large Javier Izquierdo Hernández \par}
		\vfill
		{\Large \today \par}
	\end{titlepage}

\newpage
\renewcommand{\contentsname}{Contenidos}
\tableofcontents
\newpage

\chapter{Análisis de ficheros de captura de tráfico}

\section{Cap1.cap}
\begin{enumerate}
	\item Selecciona el primer paquete que aparece en la captura, pulsando sobre la primera linea del Panel 1 (lista de paquetes). Éste quedará marcado en un color diferente:
	
	\includegraphics*[width=128mm, scale=0.5, center]{enunciado1}
	\item En el Panel 1 (lista de paquetes), para cada paquete se muestra:
	\begin{itemize}
		\item Su número de orden dentro de la captura (columna No.). El número 1 es el primer paquete capturado.
		\item Tiempo en segundos que ha pasado desde que se capturó el primer paquete (columna Time). El primer
		paquete marca el origen de tiempos, por lo que el valor de tiempo es 0.000000 segundos. El segundo
		paquete muestra 0.004014 segundos lo que significa que el segundo paquete se capturó transcurridos
		0.004014 segundos desde que se capturó el primer paquete. Y así sucesivamente.
		\item Dirección de origen del paquete (columna Source). En este caso muestra la dirección origen de nivel de
		red (dirección IP).
		\item Dirección destino del paquete (columna Destination). En este caso muestra la dirección destino de nivel
		de red (dirección IP).
		\item Protocolo de más alto nivel reconocido dentro del paquete (columna Protocol).
		\item Longitud total de la trama capturada en bytes (columna Length), sin contar el campo CRC (4 bytes).
		\item Resumen de la información más importantes contenida en los protocolos reconocidos en el paquete
		(columna Info).	
	\end{itemize}

	Con el primer paquete seleccionado, observa observa en el Panel 2 de wireshark los detalles de los protocolos
	para ese paquete. Indica qué protocolos se usan en ese primer paquete y a qué nivel de la arquitectura
	TCP/IP corresponden dichos protocolos.
	\item Teniendo seleccionado el primer paquete de la captura, en la primera pestaña (Frame) del Panel 2 se muestra
	información estadistica relativa a la captura de ese paquete. Es la única pestaña que no tiene información
	de ningún protocolo contenido en el paquete, y en general no necesitaremos consultar dicha pestaña.
	\item El resto de pestañas del Panel 2 contiene las cabeceras de los protocolos reconocidos en el paquete, empezando
	por Ethernet y siguiendo con los protocolos de niveles superiores.
	\item Teniendo seleccionado el primer paquete de la captura, despliega la pestaña que se corresponde con el
	protocolo Ethernet. Indica qué campos observas en la cabecera de Ethernet, comprueba que la longitud
	de estos campos se corresponde con lo que hemos visto en la parte de teoría. Apunta los valores de estos
	campos. Para ello, puedes seleccionar la pestaña Ethernet dentro de Wireshark y con el botón derecho del
	ratón selecciona la entrada ’Copy all visible selected tree items’ para que los campos de la cabecera Ethernet
	se queden copiados en el portapapeles y puedas pegarlos en el documento de la memoria.
	
	\includegraphics*[width=128mm,scale=0.5,center]{enunciado2}
	
	Ethernet II, Src: a2:ea:21:a9:90:5f (a2:ea:21:a9:90:5f) 
	
	Dst: 7e:24:6e:dd:8f:0e (7e:24:6e:dd:8f:0e)
	
	Destination: 7e:24:6e:dd:8f:0e (7e:24:6e:dd:8f:0e)
	
	Source: a2:ea:21:a9:90:5f (a2:ea:21:a9:90:5f)
	
	Type: IPv4 (0x0800)
	\item Pulsa sobre el campo Type de la cabecera Ethernet y observa cómo en la zona del Panel 3 que muestra el
	contenido del paquete en hexadecimal, se colorea dicho valor. Observa que wireshark interpreta el valor de
	Type 0x0800 como el código asociado al protocolo IP. ¿Qué significa?
	
	Significa que el protocolo encapsulado dentro de del campo de datos es un datagrama IP. Indica que se usara el protocolo IP
	\item Observa que en las capturas no aparecen los bytes ni el preámbulo, ni el comienzo de trama . El hardware de
	la tarjeta Ethernet elimina estos campos, pues no forman parte propiamente de la trama Etherente. Observa
	que tampoco aparece el CRC: el hardware de la tarjeta Ethernet comprueba que es correcto y lo elimina
	también de la trama. Si no fuera correcto descartaria la trama y no apareceria en la captura.
	\item Selecciona el segundo paquete y observa en el Panel 2 de wireshark los detalles de los protocolos para ese
	paquete. Indica qué protocolos se usan en ese segundo paquete y a qué nivel de la arquitectura TCP/IP
	corresponden dichos protocolos.
	
	Se utiliza el protocolo Ethernet (Nivel de enlace), IP (Nivel de red) y TCP (Nivel de transporte).
	\item Con el segundo paquete seleccionado, despliega la pestaña que se corresponde con el protocolo Ethernet.
	Indica qué campos observas en la cabecera de Ethernet. A la vista de los valores de estos campos indica si
	crees que este segundo paquete lo envía la misma máquina que envía el primer paquete.
	
	Creo que lo envía otra máquina hacia la que envió el mensaje anterior.
	\item Fíjate en la longitud del primer paquete que aparece en su columna Length del Panel 1. Dicha longitud hace referencia a la longitud de toda la trama Ethernet sin el CRC. Para calcular la longitud de toda la trama
	Ethernet habría que sumar a la columna Length de una trama los 4 bytes del CRC que no aparecen en la
	trama capturada. ¿Crees que la primera trama lleva bits de relleno en Ethernet?
	
	No lleva bits de relleno porque la cantidad de bytes es mayor a 60.
	\item Si la columna Length de la trama tuviera un valor igual a 60 bytes (longitud total de la trama igual a 64
	bytes: 60 más 4 bytes del CRC) ¿podrías decir si dicha trama tiene o no relleno?
	
	Si que tiene relleno
	\item Observa el paquete número 18. Indica qué protocolos se usan en ese paquete y a qué nivel de la arquitectura
	TCP/IP corresponden dichos protocolos.
	
	Se utiliza el protocolo Ethernet (Nivel de enlace), IP (Nivel de red) y TCP (Nivel de transporte).
	\item Observa el campo longitud de la trama Ethernet asociada al paquete número 18. Si la máquina que está en-
	viando esa información hubiese tenido más datos para enviar dentro de la trama 18, explica si hubiera podido
	incluirlos también en el campo de datos de dicha trama.
	
	No porque tiene el numero maximo de bytes.
\end{enumerate}
\section{Cap2.cap}
\begin{enumerate}
	\item Teniendo seleccionado el primer paquete de la captura, despliega la pestaña que se corresponde con el
	protocolo Ethernet. Indica qué campos observas en Ethernet. Apunta los valores de estos campos.
	
	Ethernet II, Src: Cisco251\_af:f4:54 (00:07:0d:af:f4:54), 
	
	Dst: Broadcast (ff:ff:ff:ff:ff:ff)
	
	Destination: Broadcast (ff:ff:ff:ff:ff:ff)
	
	Source: Cisco251\_af:f4:54 (00:07:0d:af:f4:54)
	
	Type: ARP (0x0806)
	
	Padding: 060104000000000201000302000005010301
	\item Fíjate en el campo Type. El valor es diferente al que viste en el fichero de captura anterior. ¿A qué protocolo se refiere este valor?
	
	Pertenece al protocolo ARP
	\item ¿Qué significa el valor del campo dirección destino Ethernet que aparece en ese primer paquete?
	
	Indica que la trama será entregada a todas las estaciones de la subred (Broadcast).
	\item Fíjate en el campo longitud de la primera trama. ¿Cuánto es la longitud total de la trama contando el CRC?
	
	Es de 64 bytes
	\item En este caso, el paquete es un mensaje del protocolo ARP que va encapsulado dentro de Ethernet. Todos
	los mensajes del protocolo ARP tienen la misma longitud, 28 bytes. La cabecera de Ethernet ocupa 14 bytes
	y el CRC 4 bytes. Por tanto la longitud total de la trama seria 46 bytes y será necesario introducir relleno
	para alcanzar la longitud de trama minima en Ethernet (64 bytes). El relleno deberia ser 18 bytes.
	\item Observa para este paquete el campo Padding. ¿Qué longitud tiene? ¿Qué crees que significa este campo?
	
	Tiene una longitud de 18 bytes, que es igual al relleno, por lo que creo que es el relleno.
\end{enumerate}
\newpage

\chapter{Generación de tráfico Ethernet y análisis de la captura de tráfico}

\includegraphics*[scale=0.5, center]{enunciado3}
\setcounter{section}{0}
\section{Paquete 1}
\begin{enumerate}[label=\alph*]
	\item Dirección Ethernet origen.
	
	Es 00:07:e9:11:67:11
	\item Dirección Ethernet destino. 
	
	Es 00:07:e9:22:67:22
	\item ¿Qué crees que se hubiera capturado en las interfaces de pc1(eth0), pc2(eth0), pc4(eth0) si hubiéramos arrancado también tcpdump en dichas interfaces? ¿Por qué?
	
	Hubiera ocurrido lo mismo, porque estan conectados a un hub.
	\item Indica qué máquinas reciben la primera trama capturada y qué máquinas la procesan y se la entregan
	al protocolo de nivel superior.
	
	El mensaje es recibido por el hub que lo envía al pc2 para que este lo procese y se lo entregue al protocolo superior
	al protocolo de nivel superior.
	\item Si la primera trama llevara como dirección destino ff:ff:ff:ff:ff:ff indica qué máquinas recibirian dicha
	trama y qué máquinas se la entregarian al protocolo de nivel superior.
	
	El hub recibiría la trama, y se la enviaría a todos los pc conectados a dicho hub.
\end{enumerate}
\section{Paquete 2}
\begin{enumerate}[label=\alph*]
	\item Dirección Ethernet origen. 
	
	Es 00:07:e9:22:67:22
	\item Dirección Ethernet destino. 
	
	Es 00:07:e9:11:67:11
	\item ¿Qué crees que se hubiera capturado en las interfaces de pc1(eth0), pc2(eth0), pc4(eth0) si hubiéramos arrancado también tcpdump en dichas interfaces? ¿Por qué?
	
	Hubiera ocurrido lo mismo, porque estan conectados a un hub.
	\item Indica qué máquinas reciben la segunda trama capturada y qué máquinas la procesan y se la entregan
	al protocolo de nivel superior.
	
	El mensaje es recibido por el hub que lo envía al pc1 para que este lo procese y se lo entregue al protocolo superior
\end{enumerate}

\chapter{Dispositivos de interconexión: hub y switch}
\section{Diferencias en el comportamiento entre un hub y un switch}

\includegraphics*[scale=0.5, center]{enunciado4}
\begin{enumerate}
	\item Indica cuál es la dirección Ethernet de pc3.
	
	La dirección Ethernet es 00:07:e9:33:67:33
	\item Comprueba que la tabla de direcciones Ethernet aprendidas en s1 está vacía (sólo tiene información de las
	interfaces locales del switch).
	\item ¿Qué crees que ocurrirá en el hub1 cuando se ejecuta el comando arping desde pc1 a pc3?
	
	El hub reenviara el mensaje a todos los ordenadores, y estos comprobaran si el mensaje va dirigido a ellos, si es así lo procesarán, y si no, lo descartarán.
	\item ¿Qué crees que ocurrirá en s1 cuando se ejecuta el comando arping desde pc1 a pc3?
	
	El switch asignará la dirección Ethernet del pc1 a un puerto en la tabla de direcciones, y enviará los mensajes igual que el hub.
	Cuando pc3 envía la respuesta al mensaje también asignará este a la tabla de direcciones, y no hará nada más porque sabe que pc1 y pc3 están conectados a través del hub juntos.
	\item Inicia una captura de tráfico en pc2 y otra en pc4 y guarda los paquetes capturados en 2 ficheros diferentes
	(p1-switch1.cap y p1-switch2.cap respectivamente). Ejecuta el comando arping -c 3 desde pc1 a pc3 (la
	opción -c 3 hace que se envíen 3 paquetes de pc1 a pc3 y se reciba una respuesta por cada envío). Observa
	la tabla de direcciones Ethernet aprendidas en s1 y explica su contenido.
	
	\includegraphics*[width=128mm, scale=0.5,center]{screenshot1}
	
	El switch s1 ha aprendido las direcciones de pc1 y pc3 como he explicado en el apartado anterior. Ahora los mensajes que vayan hacia pc1 o pc3 no serán reenviados como en un hub, sino que tendrían asignada una salida determinada hasta que pasen 5 min sin recibir uno de los dos, que en ese caso se borra su dirección de la tabla.
	\item Interrumpe las capturas de tráfico y explica justificadamente los mensajes capturados en pc2 y pc4. Relaciona
	la información de las capturas con el contenido de la tabla de direcciones aprendidas por s1.
	
	Pc2: captura los 6 mensajes porque estos son distribuidos por el hub.
	
	Pc4: solo captura el primer mensaje, porque es el único en el que la dirección de destino no estaba en la tabla.
	\item Borra la tabla de direcciones aprendidas en s1. ¿Qué crees que ocurrirá en s1 cuando se ejecuta el comando
	arping desde pc1 a pc5?
	
	El switch asignará la dirección Ethernet del pc1 a un puerto en la tabla de direcciones, y enviará los mensajes igual que el hub.
	Cuando pc5 envía la respuesta al mensaje también asignará este a la tabla de direcciones, y se lo enviará al hub de donde venía pc1.
	
	Luego, los mensaje serán dirigidos a las direcciones asignadas en la tabla.
	\item Comprueba tu suposición previa ejecutando ejecutando el comando arping -c 3 en la máquina pc1 diri-
	gido a pc5 y realizando las capturas de tráfico en pc2 (en el fichero p1-switch3.cap), pc4 (en el fichero
	p1-switch4.cap) y pc5 (en el fichero p1-switch5.cap) que muestren el tráfico intercambiado. Explica el
	contenido de la tabla de direcciones aprendidas, interrumpe las capturas y explica las direcciones aprendidas
	en relación con el tráfico capturado.
	
	\includegraphics*[width=128mm, scale=0.5, center]{screenshot2}
	
	Las direcciones aprendidas son las de pc1 y las de pc5. Como había explicado anteriormente pc2 captura todos los mensajes porque está conectado al mismo hub que pc1, pc5 captura todos los mensajes prque iban o salían de él, y pc4 solo captura el primero porque era el único en que el switch no tenía asignado la dirección de destino.
	\item Para ver cómo varía la tabla de direcciones aprendidas en un switch ejecuta el siguiente comando en s1:
	
	\textbf{\quad \quad watch -n 1 brctl showmacs s1}
	
	Este comando watch ejecuta cada segundo (-n 1) el comando brctl showmacs s1. Ejecuta arping -c 5
	desde pc1 a pc5. Explica qué es lo que ves en la tabla de direcciones aprendidas. Indica en qué momento
	desaparecen las direcciones Ethernet de pc1 y pc5. Cuando hayas terminado este apartado, puedes interrumpir
	el comando watch pulsando Ctrl+c.
	
	En la tabla de direcciones se puede ver que las direcciones de pc1 y pc5 se les ha reiniciado el tiempo de envejecimiento. El tiempo máximo es 5 min o 300 seg.
	\item Imagina que en un momento dado, la tabla de direcciones aprendidas del switch es similar a la siguiente (no
	se muestran las direcciones locales del propio switch):
	\begin{enumerate}[label=\alph*]
		\item Indica qué tramas Ethernet habrá reenviado s1 para que esa tabla sea posible.
		
		Habrá enviado tramas a pc2 y a pc5
		\item ¿Qué ocurriría si el switch recibiera una trama Ethernet de pc5 dirigida a pc3?
		
		Que las reenviaría como si fuera un hub.
		\item ¿Qué ocurriría si el switch recibiera una trama Ethernet de pc4 dirigida a pc5?
		
		Añadiría la dirección de pc4 a la tabla de direcciones y le enviaría el mensaje a la dirección que tiene guardada como pc5, reseteando el tiempo de envejecimiento de pc5.
		\item ¿Cuánto tiempo falta para que el switch elimine de la tabla de direcciones aprendidas las direcciones de
		pc2 y pc5?
		
		Faltan 74.40 segundos para que sean eliminadas.
	\end{enumerate}
	\item Supón que s1 recibe la siguiente trama Ethernet:
	
	\begin{tabular}{ |c|c|c|c| } 
		\hline
		Dir Eth. Destino & Dir Eth. Origen & Protocolo & Mensaje\\ 
		\hline
		ff:ff:ff:ff:ff:ff & dir\_Ethernet\_pc1 & Arp & ...\\ 
		\hline
	\end{tabular}
	
	Indica cómo se comportaría el switch en las siguientes situaciones:
	\begin{enumerate}[label=\alph*]
		\item El switch s1 tiene la tabla de direcciones aprendidas vaciás.
		
		Guardaría la dirección de s1 y reenviaría el mensaje a todas los pc ya que la dirección de destino es de broadcast.
		\item El switch s1 tiene aprendida la dirección Ethernet de pc1 en el puerto 1.
		
		Reiniciaría el tiempo de envejecimiento de pc1 y reenviaría el mensaje a todas los pc ya que la dirección de destino es de broadcast.
	\end{enumerate}
\end{enumerate}
\section{Influencia de los cambios de conexión fíica en la tabla de direcciones	aprendidas de un switch}

\includegraphics*[scale=0.5,center]{enunciado5}
\begin{enumerate}
	\item Ejecuta el comando arping en pc1 para que envíe un mensaje Ethernet pc1 a pc4.
	\item Consulta la tabla de direcciones aprendidas en s1 y fíjate bien en los puertos donde s1 ha aprendido.
	
	\includegraphics*[width=128mm, scale=0.5, center]{screenshot3}
	\item Ejecuta el comando que te muestra la tabla de direcciones aprendidas cada segundo, para ver cómo va a
	variar su contenido.
	
	\includegraphics*[width=128mm, scale=0.5, center]{screenshot4}
	\item Inicia una captura en pc4 guarda su contenido en el fichero p1-switch6.cap
	\item Antes de que caduquen las entradas de la tabla de direcciones aprendidas en s1 (caducan a los 5 minutos),
	apaga pc1 y borra el cable que le une al hub2. Crea un nuevo cable que una pc1 y hub1 y arranca pc1. Al
	arrancar pc1 se generan automáticamente unos mensajes de autoconfiguración que provocan que el switch
	s1 aprenda el nuevo sitio de pc1. Comprueba como se han modificado las entradas en la tabla de direcciones
	aprendidas en s1 y observa el tiempo que muestra la tabla para la entrada de pc1.
	
	El puerto de pc1 cambia de 1 a 2.
	
	\includegraphics*[width=128mm, scale= 0.5, center]{screenshot5}
	\item Interrumpe la captura y observa cómo la máquina pc1 ha generado mensajes al arrancar que han provocado
	la actualización de la tabla de direcciones aprendidas del switch.
	\item ¿Qué crees que pasaría si pc1 no hubiera generado ningún trafico automático al arrancar y en pc3 se ejecutara
	arping hacia pc1?
	
	El mensaje de pc3 no hubiera llegado al nuevo pc1 porque el switch lo reenviaría a la antigua dirección que este tiene guardada, hasta que o se eliminara la dirección de pc1 por tiempo o pc1 mandase un mensaje.
\end{enumerate}
\end{document}